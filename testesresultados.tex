\chapter{Testes e An�lise de Resultados} \label{chap:testes}

Este capitulo apresenta o ambiente de testes desenvolvido para validar o sistema, os testes efetuados e os resultados obtidos.
A se��o \ref{sec:testes_ambiente} apresenta uma ambiente de testes para valida��o do sistema, que possibilita a captura de imagens de forma automatizada. Em seguida, na se��o \ref{sec:testes_testes}, s�o apresentados os testes aos quais o sistema foi submetido e os respectivos resultados obtidos. Ao final, na se��o \ref{sec:testes_consideracoes}, s�o apresentadas algumas considera��es.


\section{Ambiente de testes} \label{sec:testes_ambiente}

Inicialmente para validar o sistema proposto � necess�rio a defini��o dos objetos reconhecidos. Com o objetivo de possibilitar a valida��o do sistema por�m sem comprometer a aplica��o real do projeto utiliza-se os objetos dispostos na figura \ref{}.




Para validar o sistema desenvolvido � necess�rio um ambiente que permita a captura de imagens de objetos em diversas orienta��es angulares. � necess�rio tamb�m a defini��o de um conjunto de objetos que ser�o reconhecidos 

\section{Testes} \label{sec:testes_testes}
\section{Considera��es} \label{sec:testes_consideracoes}

Nesse cap�tulo apresentou-se um ambiente de testes que possibilitou a valida��o do sistema apresentado no capitulo \ref{chap:desenv}.
Utilizando o ambiente forma executados....
obtendo...